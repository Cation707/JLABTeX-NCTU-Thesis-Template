\newpage
\thispagestyle{plain}
	\addcontentsline{toc}{chapter}{摘  要}
	%\pagenumbering{Roman}
\begin{center}
	\TSzTwenty\selectfont
	%\begin{singlespace}
		\textbf{毫米波頻帶之單一使用者多輸入多輸出混合式預編碼與結合器設計} \\[0.5cm]
	%\end{singlespace}
	\TSzFourteen\selectfont
%	\begin{singlespace}    
		\begin{tabular}{r l r l}
			學生: & 李財華& \hspace{4cm} 指導教授: & 周世傑、劉志尉 教授 \\
		\end{tabular}
%	\end{singlespace}
    \\[0.5cm]
	\TSzFourteen\selectfont
	國立交通大學 \\[0.5cm] 
	電子研究所 \\[0.5cm] 
	碩士班 \\[0.5cm]
	\CJKfakebold{摘要} \\[0.5cm]
\end{center}

%\normalsize 
\TSzTwelveThirty\selectfont
在大 AI、ML 時代,自己寫論文已經不再是個有效率的做法,因此我們提出了一套基於卷積神經網路的論文自動生成技術。

關鍵字:卷積神經網路、機器學習 在大 AI、ML 時代,自己寫論文已經不再是個有效率的做法,因此我們提出了一套基於卷積神經網路的論文自動生成技術。 
關鍵字:卷積神經網路、機器學習 在大 AI、ML 時代,自己寫論文已經不再是個有效率的做法,因此我們提出了一套基於卷積神經網路的論文自動生成技術。 
關鍵字:卷積神經網路、機器學習 在大 AI、ML 時代,自己寫論文已經不再是個有效率的做法,因此我們提出了一套基於卷積神經網路的論文自動生成技術。 
關鍵字:卷積神經網路、機器學習在大 AI、ML 時代,自己寫論文已經不再是個有效率的做法,因此我們提出了一套基於卷積神經網路的論文自動生成技術。 
關鍵字:卷積神經網路、機器學習 在大 AI、ML 時代,自己寫論文已經不再是個有效率的做法,因此我們提出了一套基於卷積神經網路的論文自動生成技術。 
關鍵字:卷積神經網路、機器學習 在大 AI、ML 時代,自己寫論文已經不再是個有效率的做法,因此我們提出了一套基於卷積神經網路的論文自動生成技術。 
關鍵字:卷積神經網路、機器學習 在大 AI、ML 時代,自己寫論文已經不再是個有效率的做法,因此我們提出了一套基於卷積神經網路的論文自動生成技術。 
關鍵字:卷積神經網路、機器學習在大 AI、ML 時代,自己寫論文已經不再是個有效率的做法,因此我們提出了一套基於卷積神經網路的論文自動生成技。
關鍵字:卷積神經網路、機器學習 在大 AI、ML 時代,自己寫論文已經不再是個有效率的做法,因此我們提出了一套基於卷積神經網路的論文自動生成技術。 
關鍵字:卷積神經網路、機器學習 在大 AI、ML 時代,自己寫論文已經不再是個有效率的做法,因此我們提出了一套基於卷積神經網路的論文自動生成技術。 
關鍵字:卷積神經網路、機器學習 在大 AI、ML 時代,自己寫論文已經不再是個有效率的做法,因此我們提出了一套基於卷積神經網路的論文自動生成技術。 
關鍵字:卷積神經網路、機器學習在大 AI、ML 時代,自己寫論文已經不再是個有效率的做法,因此我們提出了一套基於卷積神經網路的論文自動生成技術。
關鍵字:卷積神經網路、機器學習 在大 AI、ML 時代,自己寫論文已經不再是個有效率的做法,因此我們提出了一套基於卷積神經網路的論文自動生成技術。 
關鍵字:卷積神經網路、機器學習 在大 AI、ML 時代,自己寫論文已經不再是個有效率的做法,因此我們提出了一套基於卷積神經網路的論文自動生成技術。 
關鍵字:卷積神經網路、機器學習 在大 AI、ML 時代,自己寫論文已經不再是個有效率的做法,因此我們提出了一套基於卷積神經網路的論文自動生成技術。 
關鍵字:卷積神經網路、機器學習在大 AI、ML 時代,自己寫論文已經不再是個有效率的做法,因此我們提出了一套基於卷積神經網路的論文自動生成技術。
%\pagenumbering{Roman}

